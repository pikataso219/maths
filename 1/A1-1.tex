\documentclass[b5paper,12pt]{jarticle}
\usepackage{ascmac}
\usepackage{fancybox}

\begin{document}

\section*{リーマンゼータ関数のだね例}
\section{チェビシェフ多項式}
\begin{itembox}[l]{1-1}
   (1) $\cos5\theta $=$f(\cos\theta )$ を満たす多項式$f(x)$ を求めよ.  
   \par
   (2) $\cos\frac{\pi}{10}\cos\frac{3\pi}{10}\cos\frac{7\pi}{10}\cos\frac{9\pi}{10}=\frac{5}{16}$
\par \hfill \footnotesize(京都大)
\end{itembox}

\doublebox{解答} \par
加法定理から、$\cos5\theta = \cos(3\theta+2\theta)$
\par
$s > 1$ において級数
\begin{equation}
 \zeta(s) = \sum_{n=1}^{\infty} \frac{1}{n^s}
\end{equation}
は収束する.$\zeta(s)$ を \textbf{リーマンのゼータ関数} と呼ぶ.

オイラーは
\begin{equation}
 \zeta(2) = \sum_{n=1}^{\infty} \frac{1}{n^2} = \frac{\pi^2}{6}
\end{equation}
であることを発見した.

\end{document}