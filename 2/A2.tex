\documentclass[b5paper,12pt]{jarticle}
\usepackage{ascmac}
\usepackage{fancybox}
\begin{document}
\begin{itembox}[l]{2-1}
    $\bigtriangleup ABCの外心(外接円の中心)Oが三角形の内部にあるとし、$
    \par
    $\alpha、\beta 、\gamma は$
    \par
    \centerline{$\alpha \overrightarrow{OA}+ \beta \overrightarrow{OB}+ \gamma \overrightarrow{OC} = \overrightarrow{0}$}
    \par
    $を満たす正数であるとする. また、直線OA、OB、OCがそれぞれ辺BC、CA、ABと交わる点をA'、B'、C'とする.$
    \par
    $(1) \overrightarrow{OA}、 \alpha 、 \beta 、 \gamma を用いて\overrightarrow{OA'} を表せ.   $
    \par
    $(2) \bigtriangleup A'B'C'の外心がOに一致すれば\alpha =\beta =\gamma であることを示せ.$
\end{itembox}
\begin{itembox}[l]{2-2}
    $座標平面上で、一つの円が放物線y=x^2に右側から接し、かつx軸に上から接している. 放物線との接点Aのx座標をa(>0)とするとき、円の中心Cの座標を求めよ.$
    \par
    $ただし、円と放物線がある点で接するとは、その点で両者が交わり、かつその点における両者の接線が一致することをいう.$
\end{itembox}
\begin{itembox}[l]{2-3}
    $円C:x^2+y^2=r^2の外部の点A(x1、y1)から円Cに引いた二本の接線と円Cとの接点をQ、Rとするとき、直線QRの方程式を求めよ.$
\end{itembox}
\begin{itembox}[l]{2-4}
    $xyz平面上に三点A(1,0,1),B(3,1,-1),C(6,4,-1)がある. 点Cの直線ABに関する対称点の座標を求めよ.$
\end{itembox}
\end{document}