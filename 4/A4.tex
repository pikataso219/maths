\documentclass[b5paper,12pt]{jarticle}
\usepackage{ascmac}
\usepackage{fancybox}
\begin{document}
\begin{itembox}[l]{4-1}
    $座標空間に4点A(2,1,0)B(1,0,1)C(0,1,2)D(1,3,7)がある.3点A、B、Cを通る平面に関して点Dと対称な点をEとするとき、点Eの座標を求めよ.$
\end{itembox}
\begin{itembox}[l]{4-2}
    $Oを原点とするxyz空間内に5点 $
    \par
    \centerline {$A(-1,0,0)B(0,2,0)C(0,0,1)D(0,0,2)E(0,0,4)$}
    \par
    $をとる.中心がD、半径が2の球面をSとし、3点A、B、Cの定める平面を\alpha とする.Sが\alpha と交わってできる図形をFとする.点PはF上を動く点とし、直線EPとxy平面との交点をQ(s,t,0)とする.このとき、s、tが満たす方程式を求めよ.$
\end{itembox}
\begin{itembox}[l]{4-3}
    $xyz空間内の平面z=0の上にx^2+y^2=25により定まる円Cがあり、平面z=4の上にx=1により定まるy軸に平行な直線lがある.$
    \par
    $(1) 点P(6,8,15)からC上の点への距離の最小値を求めよ.$
    \par
    $(2) C上の点で、l上の点への距離の最小値が5であるものをすべて求めよ.$
\end{itembox}
\end{document}