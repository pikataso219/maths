\documentclass[b5paper,12pt]{jarticle}
\usepackage{ascmac}
\usepackage{fancybox}
\begin{document}
\begin{itembox}[l]{5-1}
    $(a,b)はxy平面上の点とする.点(a,b)から曲線y=x^3-xに接線がちょうど2本だけひけ、この2本の接線が直交するものとする.このときの(a,b)を求めよ.$
\end{itembox}
\begin{itembox}[l]{5-2}
    $関数f(x)が$
    \par
    \centerline{$f(x)=x^2-x\int_{0}^{2} |f(x)|\,dx $}
    \par
$を満たしているとする.このとき、f(x)を求めよ.$
\end{itembox}
\begin{itembox}[l]{5-3}
    $関数f(x)=x^4-2x^2+xについて、次の問いに答えよ.$
    \par
    $(1)曲線y=f(x)と2点で接する直線の方程式を求めよ.$
    \par
    $(2)曲線y=f(x)と(1)で求めた直線で囲まれた領域の面積を求めよ.$
\end{itembox}
\begin{itembox}[l]{5-4}
    $0\leq k\leq1 を満たす実数kに対して、xy平面上に次の連立不等式で表される3つの領域D、E、Fを考える. $
    \par
    $Dは連立不等式y\geq x^2、 y\leq kxで表される領域$
    \par
    $Eは連立不等式y\leq x^2、 y\geq kxで表される領域$
    \par
    $Fは連立不等式y\leq -x^2+2x、 y\geq kxで表される領域$
    \par
    $(1)領域D\cup (E\cap F) の面積m(k)を求めよ.$
    \par
    $(2)(1)で求めた面積m(k)を最小にするkの値と、最小値を求めよ.$
\end{itembox}
\end{document}