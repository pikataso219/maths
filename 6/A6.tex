\documentclass[b5paper,12pt]{jarticle}
\usepackage{ascmac}
\usepackage{fancybox}
\begin{document}
\begin{itembox}[l]{6-1}
    $nを自然数とするとき、以下の問いに答えよ.$
    \par
    $(1)  n\geq 3とする.1からnまでの自然数の中から連続しない相違なる2つの数を選ぶ選び方は何通りあるか求めよ.$
    \par
    $(2) n\geq 5とする.1からnまでの自然数の中からどの二つも連続しない相違なる3つの数を選ぶ選び方は何通りあるか求めよ.$
\end{itembox}
\begin{itembox}[l]{6-2}
    $Nを2以上の整数とする.1\leq a <b<c\leq 2Nを満たし、a、b、cを3辺の長さとする三角形が存在するような整数の組(a、b、c)の個数をS_N とする.$
    \par
    $(1)S_3を求めよ.$
    \par
    $(2)S_NをNを用いて表せ.$
\end{itembox}
\begin{itembox}[l]{6-3}
    $nを正の整数とし、n個のボールを3つの箱に分けて入れる問題を考える.ただし、1個のボールも入らない箱があってもよいものとする.以下に述べる4つの場合について、それぞれ相違なる入れ方の総和を求めたい.$
    \par
    $(1)1からnまで異なる番号のついたn個のボールを、A、B、Cと区別された3つの箱に入れる場合、その入れ方は全部で何通りあるか.$
    \par
    $(2)互いに区別のつかないn個のボールを、A、B、Cと区別された3つの箱に入れる場合、その入れ方は全部で何通りあるか.$
    \par
    $(3)1からnまでの異なる番号のついたn個のボールを、区別のつかない3つの箱に入れる場合、その入れ方は全部で何通りあるか.$
    \par
    $(4)nが6の倍数6mであるとき、n個の互いに区別のつかないボールを、区別のつかない3つの箱に入れる場合、その入れ方は全部で何通りあるか.$
\end{itembox}
\end{document}