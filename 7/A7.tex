\documentclass[b5paper,12pt]{jarticle}
\usepackage{ascmac}
\usepackage{fancybox}
\begin{document}
\begin{itembox}[l]{7-1}
    $先生と三人の生徒A、B、Cがおり、玉の入った箱がある.$
    \par
    $箱の中には最初、赤玉3個、白玉7個、全部で10個の玉が入っている. 先生がサイコロをふって、1の目が出たらAが、2または3の目がでたらBが、その他の目が出たらCが箱の中から1つだけ玉を取り出す操作を行う.取り出した玉は箱の中には戻さず、取り出した生徒のものとする.この操作を続けて行うものとして、以下の問いに答えよ.$
    \par
    $(1)2回目の操作が終わった時、Aが2個の赤玉を手に入れている確率を求めよ.$
    \par
    $(2)2回目の操作が終わった時、Bが少なくとも1個の赤玉を手に入れている確率を求めよ.$
    \par
    $(3)3回目の操作で、Cが赤玉を取り出す確率を求めよ.$
\end{itembox}
\begin{itembox}[l]{7-2}
    $数字の2を書いた玉が1個、数字の1を書いた玉が3個、数字の0を書いた玉が4個あり、これら合計8個の玉が袋に入っている.この状態の袋から1度に1個づつ玉を取り出し、取り出した玉は袋に戻さないものとする.玉を8度取り出すとき、次の条件が満たされる確率を求めよ.$
    \par
    $条件: 全てのn=1、2、...、8に対して、1個目からn個目までの玉に書かれた数字の合計はn以下である.$
\end{itembox}
\begin{itembox}[l]{7-3}
    $点Pが次のルール\mathrm{(i)} 、\mathrm{(ii)}に従って数直線上を移動するものとする.$
    \par
    $\mathrm{(i)}1、2、3、4、5、6の目が同じ割合で出るサイコロを振り、出た目の数をkとする.Pの座標aについて、a>0ならば座標a-kの点に移動し、a<0ならば座標a+kの点に移動する.$
    \par
    $\mathrm{(ii)}原点に移動したら終了し、そうでなければ\mathrm{(i)}を繰り返す.$
    \par
    $(1)Pの座標が1、2、...、6のいずれかであるとき、ちょうどm回のサイコロを振って原点で終了する確率を求めよ.$
    \par
    $(2)Pの座標が8であるとき、ちょうどn回のサイコロを振って原点で終了する確率を求めよ.$
\end{itembox}
\begin{itembox}[l]{7-4}
    $白黒2種類のカードがたくさんある.そのうちk枚のカードを手元に持っているとき、次の操作(A)を考える.$
    \par
    $(A): 手持ちのk枚の中から1枚を、当確率 \frac{1}{k} で選び出し、それを違う色のカードにとりかえる.$
    \par
    $(1)最初に白2枚、黒2枚、合計4枚のカードをもっているとき、操作(A)をn回繰り返した後に初めて、4枚とも同じ色のカードになる確率を求めよ.$
    \par
    $(2)最初に白3枚、黒3枚、合計6枚のカードを持っているとき、操作(A)をn回繰り返した後に初めて、6枚とも同じ色のカードになる確率を求めよ.$
\end{itembox}
\begin{itembox}[l]{7-5}
    $最初の試行で3枚の硬貨を同時に投げ、裏が出た硬貨を取り除く.次の試行で残った硬貨を同時に投げ、裏が出た硬貨を取り除く.以下この試行をすべての硬貨が取り除かれるまで繰り返す.このとき、試行がn回目で終了する確率を求めよ.$
\end{itembox}
\begin{itembox}[l]{7-6}
    $どの目も出る確率が\frac{1}{6}のサイコロを1つ用意し、次のように左から順に文字を書く.$
    \par
    $サイコロを投げ、出た目が1、2、3のときは文字列AAを書き、4のときは文字Bを、5のときは文字Cを、6のときは文字Dを書く.さらに繰り返しサイコロを投げ、同じ規則に従って、AA、B、C、Dをすでにある文字列の右側につなげて書いていく.$
    \par
    $たとえば、サイコロを5回投げ、その出た目が順に2、5、6、3、4であったとすると、得られる文字列は、$
    \par
    $AACDAAB$
    \par
    $となる.このとき、左から4番目の文字はD、5番目の文字はAである.$
    \par
    $(1)nを正の整数とする.n回サイコロを投げ、文字列を作るとき、文字列の左からn番目の文字がAとなる確率を求めよ.$
    \par
    $(2)nを2以上の整数とする.n回サイコロを投げ、文字列を作るとき、文字列の左からn-1番目の文字がAで、かるn番目の文字がBとなる確率を求めよ.$
\end{itembox}
\begin{itembox}[l]{7-7}
    $水戸黄門、助さん、格さん、弥七、お銀、八兵衛の6人が左から右にこの順番で1列に並んで座っている.6人が席を入れ換える.どの並びからも同様の確からしさで起こるものとする.このとき、最初と同じ席に座る人がいない確率を求めよ.$
\end{itembox}
\begin{itembox}[l]{7-8}
    $さいころをn回振り、出る目の数n個の積をX_n、出る目の数のn個の和をY_nとする.$
    \par
    $(1)X_nが3の倍数である確率p_nを求めよ.$
    \par
    $(2)Y_nが3の倍数である確率q_nを求めよ.$
    \par
    $(3)X_nが3の倍数、かつY_nが3の倍数である確率r_nを求めよ.$
\end{itembox}
\begin{itembox}[l]{7-9}
    $4つの箱があり、そのうちの2つに当たりくじが入っている.$
    \par
    $(1)太郎が先に1つの箱を選び、次に花子が残りから1つ選ぶ.このとき、花子が当たりの箱を選ぶ確率を求めよ.$
    \par
    $(2)太郎が先に1つの箱を選んでまだ開けないうちに、どれに当たりのくじが入っているかを知らない司会者が別の箱を1つ開けたところはずれであった.このとき、太郎の箱が当たりである確率を求めよ.また、残りの2つの箱から花子が当たりの箱を選ぶ確率を求めよ.$
    \par
    $(3)太郎が先に1つの箱を選んでまだ開けないうちに、どれに当たりのくじが入っているかを知っている司会者がはずれの箱を1つ開けた.このとき、太郎の箱が当たりである確率を求めよ.また、残りの2つの箱から花子が当たりの箱を選ぶ確率を求めよ.$
\end{itembox}
\begin{itembox}[l]{7-10}
    $nを3以上の整数とする.$
    \par
    $(1)\sum_{k = 1}^{n} k(k-1) {}_n \mathrm{C}_k を求めよ.$
    \par
    $(2)\sum_{k = 1}^{n} k^2{}_n \mathrm{C}_k を求めよ.$
\end{itembox}
\begin{itembox}[l]{7-11}
    $nを3以上の自然数とする.スイッチを入れると等確率で赤色または青色に輝く電球が横一列にn個並んでいる.これらのn個の電球のスイッチを同時に入れたあと、左から電球の色を見ていき、色の変化の回数を調べる.$
    \par
    $(1)赤青...青、赤赤青...青、...のように左端が赤色で色の変化がちょうど1回起きる確率を求めよ.$
    \par
    $(2)色の変化が少なくとも2回起きる確率を求めよ.$
    \par
    $(3)色の変化がちょうどm回(0\leq m\leq n-1)起きる確率p_mを求めよ.$
    \par
    $(4)色の変化の回数の期待値を求めよ.$
\end{itembox}
\end{document}