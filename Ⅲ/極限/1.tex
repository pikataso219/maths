\documentclass[b5paper,12pt]{jarticle}
\usepackage{ascmac}
\usepackage{fancybox}
\usepackage{amsmath}
\usepackage{ulem}
\begin{document}
    時間の都合上、FGや青チャートに載っている簡単なものは軽くのみ触れます。しかし、この中にもよく出る、重要なものがあるのでそこを重点的に載せます。
    \vskip \baselineskip
    極限の9割は不定形をどう処理するかということです。\newline 一番簡単なものは式変形でどうにかなります。そのへんは網羅系参考書で頑張ってもらうとして、次に「\textbf{公式}」があります。
    \vskip \baselineskip
    \begin{itembox}{極限の重要公式}
    $\uline{1.三角関数の極限}$
    \vskip.5 \baselineskip
    $\lim\limits_{\theta  \to 0} \dfrac{\sin\theta}{\theta} =1 $
    \vskip.5 \baselineskip
    $\theta\rightarrow 0に注意してください。$
    \vskip.5 \baselineskip
    $\lim\limits_{\theta \to 0} \dfrac{1-\cos\theta}{\theta^2} =\dfrac{1}{2}$
    \vskip.5 \baselineskip
    $\lim\limits_{\theta \to 0} \dfrac{\tan\theta}{\theta} =1 $
    \vskip.5 \baselineskip
    $単なる変形ですが、よく出るので余力があれば覚えるべきです。$
    \vskip \baselineskip
    $2. \textbf{e}の定義$
    \vskip.5 \baselineskip
    $曲線y=f(x)=a^x上の点(0,1)における接線の傾きは、$
    \vskip.5 \baselineskip
    $f'(0)=\lim\limits_{h \to 0} \dfrac{f(0+h)-f(0)}{h} = \lim\limits_{h \to 0} \dfrac{a^h-1}{h}$
\end{itembox}
\end{document}